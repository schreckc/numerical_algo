

\lstset{
  language=Algo,
  basicstyle=\sffamily,
  columns=fullflexible,
  mathescape
}

\begin{center}
{\large\textbf{Homework 1: numerics and error analysis}}\\
Due Monday, March 6.
\end{center}

\noindent\makebox[\linewidth]{\rule{\linewidth}{0.6pt}}
 
\section{Instructions}

\begin{itemize}
\item The first three exercises are designed to review basic mathematics that we will be using for the rest of the course. It is strongly recommended to read the chapter 1 of the textbook an make sure you have a good grasp on the knowledge and techniques described there.
\item The assignments must be dropped off in the T.A. mail box: Office Building West, 2nd floor, left of the stairs, top-left box (Camille Schreck). The deadline is Monday, March 6 at midnight. 
\item The assignments may be handwritten or typed. You can code in any programming language. The code itself will not be review. 
\item The textbook is \emph{Numerical Algorithms} from Justin Solomon.\\ \url{https://people.csail.mit.edu/jsolomon/share/book/numerical_book.pdf}.
\end{itemize}



\noindent\makebox[\linewidth]{\rule{\linewidth}{0.6pt}}

\section{Exercise 1: matrices \normalsize \textnormal(10 points)}

Suppose the rows of $A \in \mathds{R}^{m \times n}$ are given by the transposes of $\vec{r}_1, \hdots , \vec{r}_m \in  \mathds{R}^n$ and
the columns of $A \in \mathds{R}^{m \times n}$ are given by  $\vec{c}_1, \hdots , \vec{c}_n \in  \mathds{R}^m$. That is,
$$ A =
\left(\begin{matrix}
  - & \vec{r}_1^\top & - \\
   & \vdots &  \\
  - & \vec{r}_m^\top & -
\end{matrix} \right) =
\left(\begin{matrix}
  | &  & | \\
 \vec{c}_1 & \hdots &   \vec{c}_n\\
  | &  & |
\end{matrix} \right)
$$

Give expressions for the elements of $A^\top A$ and $AA^\top$ in terms of these vectors.


\begin{correction}
$$ A^\top = \left(\begin{matrix}
  | &  & | \\
 \vec{r}_1 & \hdots &   \vec{r}_m\\
  | &  & |
  \end{matrix} \right)
  =
  \left(\begin{matrix}
  - & \vec{c}_1^\top & - \\
   & \vdots &  \\
  - & \vec{c}_n^\top & -
  \end{matrix} \right)$$
  So,
$$A^\top A =   \left(\begin{matrix}
  - & \vec{c}_1^\top & - \\
   & \vdots &  \\
  - & \vec{c}_n^\top & -
  \end{matrix} \right) \cdot
  \left(\begin{matrix}
  | &  & | \\
 \vec{c}_1 & \hdots &   \vec{c}_n\\
  | &  & |
  \end{matrix} \right) =  \left( \begin{matrix} \vec{c}_1 \cdot \vec{c}_1 & \hdots & \vec{c}_1 \cdot \vec{c}_n \\
    \vdots & & \vdots \\
    \vec{c}_n \cdot \vec{c}_1 & \hdots & \vec{c}_n \cdot \vec{c}_n \\
    \end{matrix} \right)
  \text{.}$$
$A^\top A$ is a matrix of size $n \times n$ and each element $(A^\top A)_{i, j} = c_i \cdot c_j$.
  $$AA^\top = \left(\begin{matrix}
  - & \vec{r}_1^\top & - \\
   & \vdots &  \\
  - & \vec{r}_m^\top & -
  \end{matrix} \right) \cdot
  \left(\begin{matrix}
  | &  & | \\
 \vec{r}_1 & \hdots &   \vec{r}_m\\
  | &  & |
  \end{matrix} \right) =  \left( \begin{matrix} \vec{r}_1 \cdot \vec{r}_1 & \hdots & \vec{r}_1 \cdot \vec{r}_m \\
    \vdots & & \vdots \\
    \vec{r}_m \cdot \vec{r}_1 & \hdots & \vec{r}_m \cdot \vec{r}_m \\
    \end{matrix} \right)
  $$ 
  $AA^\top$ is a matrix of size $m \times m$ and each element $(AA^\top)_{i, j} = r_i \cdot r_j$.  
\end{correction}


\section{Exercise 2: optimization  \normalsize \textnormal(25 points)} 

\begin{itemize}
\item[(1)] Give a linear system of equations satisfied by minima of the energy $f(\vec{x}) = ||A\vec{x}-\vec{b}||_2$ with respect to $\vec{x}$, for $\vec{x} \in \mathds{R}^n$, $A \in \mathds{R}^{m \times n}$, and $\vec{b} \in \mathds{R}^m$.
\begin{correction}
     The critical points of $f(\vec{x})$ are reached for $\frac{\partial f}{\partial \vec{x}} = 0$.\newline
    \newline
    If $Ax - b = 0$ the the critical points are the solutions of this system.\newline
    \newline
    In the other cases, we need to compute $\frac{\partial f}{\partial \vec{x}}$. \\
    Let $g(\vec{u}) = ||\vec{u}||_2$. Then $\frac{\partial g}{\partial \vec{u}} = \frac{\vec{u}}{||u||_2}^\top$ for $\vec{u} \neq \vec{0}$. \\
    And let $\vec{u}(\vec{x}) = A\vec{x} - \vec{b}$. Then $\frac{\partial \vec{u}}{\partial \vec{x}} = A$.\\
    So $f(\vec{x}) = g(\vec{u}(\vec{x}))$ and
    $$ \frac{\partial f}{\partial \vec{x}} = \frac{\partial g}{\partial \vec{u}} \frac{\partial \vec{u}}{\partial \vec{x}} = (A\vec{x} + \vec{b})A = \left(A^\top(A\vec{x} - \vec{b})\right)^\top \text{.}$$
    So critical points of $f$ satisfy $A^\top (A\vec{x}  -\vec{b}) = \vec{0}$. \newline 
    \newline
    In both case minima of energy satisfy $A^\top A\vec{x} = A^\top\vec{b}$.  
\end{correction}
 
\item[(2)] Fix some vector $\vec{a}\in \mathds{R}^n\backslash \{\vec{0}\}$ and define $f(\vec{x}) = \vec{a} \cdot \vec{x}$. Give an expression for the maximum of $f(\vec{x})$ subject to $||\vec{x}||_2 = 1$.
  \begin{correction}
    Let's define $\Lambda(\vec{x}, \lambda) = f(\vec{x}) - \lambda(||\vec{x}||_2 - 1)$.
    $$\frac{\partial \Lambda}{\partial \vec{x}} = \vec{a} - \lambda \frac{\vec{x}}{||\vec{x}||_2} \text{\hspace{1cm} and \hspace{1cm}} \frac{\partial \Lambda}{\partial \lambda} = ||\vec{x}||_2 - 1 \text{.}$$
    Critical points of $\Lambda$ then satisfied $\vec{a} = \lambda \frac{\vec{x}}{||\vec{x}||_2}$ and $||\vec{x}||_2 = 1$.
    By combining these two conditions we obtain $\vec{a} = \lambda \vec{x}$. So $\lambda = \pm ||\vec{a}||_2$ and then the critical points are: 
    $$\vec{x}_{c_1} = \frac{\vec{a}}{||\vec{a}||_2} \text{\hspace{0.4cm} and \hspace{0.4cm}} \vec{x}_{c_2} = -\frac{\vec{a}}{||\vec{a}||_2}\text{.}$$
    So the maximum of $f(\vec{x})$ subject to $||\vec{x}||_2 = 1$ if obtained for $\vec{x}_{c_1} = \frac{\vec{a}}{||\vec{a}||_2}$ and $f(\vec{x}_{c_1}) = ||\vec{a}||_2$.
\end{correction}

\end{itemize}


\section{Exercise 3: idempotent matrices  \normalsize \textnormal(15 points)}

A matrix  $A \in \mathds{R}^{n \times n}$ is idempotent if it satisfies $A^2 = A$.

\begin{itemize}
\item[(1)] Suppose $B \in \mathds{R}^{m \times k}$ is constructed so that $B^\top B$ is invertible. Show that the matrix $B(B^\top B)^{-1} B^\top$ is idempotent.

  \begin{correction}
$$(B(B^\top B)^{-1} B^\top)^2 = (B(B^\top B)^{-1} B^\top)(B(B^\top B)^{-1} B^\top) $$
$$= B(B^\top B)^{-1} (B^\top B)(B^\top B)^{-1} B^\top$$
$$              =  B I_{k \times k}(B^\top B)^{-1} B^\top$$
$$= B(B^\top B)^{-1} B^\top$$
\end{correction}
\item[(2)] If $A$ is idempotent, show that $I_{n\times n} - A$ is also idempotent.
 \begin{correction}
\begin{multline*}
 (I_{n\times n} - A)^2 = I_{n\times n}^2 - 2I_{n\times n}A + A^2 = I_{n\times n} - 2A + A = I_{n\times n} - A
      \end{multline*}                             
\end{correction}

\item[(3)] If $A$ is idempotent, show that $\frac{1}{2} I_{n\times n} - A$ is invertible and give an expression for its inverse.
\begin{correction}
\begin{multline*}
 (\frac{1}{2}I_{n\times n} - A)^2 = \frac{1}{4}I_{n\times n}^2 - 2\frac{1}{2}I_{n\times n}A + A^2 = \frac{1}{4}I_{n\times n} - A + A = \frac{1}{4}I_{n\times n}
      \end{multline*}
      So $\frac{1}{2}I_{n\times n} - A$ is invertible and its inverse is $4(\frac{1}{2}I_{n\times n} - A) = 2I_{n\times n} - 4A$.
\end{correction}
\item[(4)] Suppose $A$ is idempotent and that we are given $\vec{x} \neq \vec{0}$ and $\lambda \in  \mathds{R}$ satisfying $A\vec{x} = \lambda\vec{x}$. Show that $\lambda \in \{0, 1\}$.
   \begin{correction}
   If $A\vec{x} = \lambda\vec{x}$ then
   $$A^2\vec{x} = \lambda A\vec{x}$$
   $$A\vec{x} = \lambda A\vec{x}$$
   $$\lambda\vec{x}  = \lambda^2\vec{x}$$
   So $\lambda = \lambda^2$ which means that $\lambda \in \{0, 1\}$.
\end{correction}   
\end{itemize}

\section{Exercise 4: Sum Bad Pi  \normalsize \textnormal(50 points)}

One of the lousier ways to compute $\pi$ is by using the Gregory series
$$ G_n = 4 \sum_{k=1}^n\frac{(-1)^{k-1}}{2k-1} \xrightarrow{n\to \infty} \pi = 3.14159265358979323846264338327950288419716939937510582\hdots$$
Its convergence is slow and, well, unusual. In this question, you will explore how many correct digits can you compute of $\pi$ using this formula --the matching digits may not all be contiguous!

\begin{itemize}
\item[(1)] Write a simple program, using double-precision floating point arithmetic, that computes $G_n$. Evaluate $G_n$ for $n = 10000, 100000, 1000000$ and $5000000$. Compare your results to $\pi$. How many digits match?
\item[(2)] Next,  implement  the  more  careful  Kahan  Sum  algorithm  from  the  textbook (\textsection 2.3.2), and compare your results for the $n$ values used previously. How many digits match?
\item[(3)]One might be bothered by the alternating nature of this series.  Consider the modified non-alternating series where each term is a positive number,
$$ G_n = 4 \sum_{k=1\text{, $k$ odd}}^n\left(\frac{1}{2k-1} - \frac{1}{2(k+1)-1}\right) = 2\sum_{k=1\text{, $k$ odd}}^n\frac{1}{(k - \frac{1}{2})(k + \frac{1}{2})} \text{\hspace{2cm}($n$ even)}.$$  
  Compare your results for this summation (with/without Kahan Sum) to the previous values. Is this approach better or worse? Why ?
\item[(4)] Plot the output of two methods and compare them.
\end{itemize}


