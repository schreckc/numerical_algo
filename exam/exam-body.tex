

\lstset{
  language=Algo,
  basicstyle=\sffamily,
  columns=fullflexible,
  mathescape
}

\begin{center}
{\large\textbf{Examination}}\\
\end{center}

\noindent\makebox[\linewidth]{\rule{\linewidth}{0.6pt}}
 
\section{Instructions}

\begin{itemize}
\item
\item
\end{itemize}



\noindent\makebox[\linewidth]{\rule{\linewidth}{0.6pt}}

\section{Exercise 1: }

\section{Exercise 2: Polar Decomposition}

In this problem we will add one more matrix factorization to our linear algebra toolbox and derive an algorithm by N. Higham for its computation.  The decomposition  is  used  in  animation  applications  interpolating  between  motions  of  a  rigid  object while projecting out undesirable shearing artifacts.

{\it\textbf{Definition and reminders:}
\begin{itemize}
\item Othogonal matrix: In linear algebra, an orthogonal matrix or real orthogonal matrix ${\bf Q}$ is a square matrix with real entries whose columns and rows are orthogonal unit vectors (i.e., orthonormal vectors), i.e.
    $${\bf Q}^\top {\bf Q} = {\bf Q \: Q}^\top = {\bf I},$$
where ${\bf I}$ is the identity matrix.\\
This leads to the equivalent characterization: a matrix ${\bf Q}$ is orthogonal if its transpose is equal to its inverse:
    $${\bf Q}^\top = {\bf Q}^{-1}.$$
\item Positive semidefinite  ??
\item Symmetric ??
\item SVD: the singular value decomposition of an $m \times n$ real matrix ${\bf A}$ is a factorization of the form  ${\bf U \: \Sigma \: V^\top}$, where {\bf U} is an $m \times m$ orthgonal matrix, ${\bf \Sigma}$ is a $m \times n$ rectangular diagonal matrix with non-negative real numbers on the diagonal, and {\bf V} is an $n \times n$ orthgonal matrix.
\end{itemize}
}
\subsection{} Consider two orthogonal matrices ${\bf U}$ and ${\bf V}$. Show that ${\bf U\:V}$ is an othogonal matrix.

\subsection{} Show that any matrix  ${\bf A} \in R\mathds{R}^{n \times n}$ can be factored ${\bf A} = {\bf W\:P}$, where  ${\bf W}$ is orthogonal and  ${\bf P}$ is symmetric and positive semidefinite. This factorization is known as the polar decomposition.\\
\textit{Hint: Write the SVD of ${\bf A} = {\bf U \: \Sigma \: V^\top}$ and show that ${\bf V \: \Sigma \: V}^\top$ is positive semidefinite.}

\subsection{}\label{scheme} The polar decomposition of an invertible ${\bf A} \in R\mathds{R}^{n \times n}$ can be computed using a simple iterative scheme:
$$ {\bf X}_0 = {\bf A} \text{\hspace{2cm}} {\bf X}_{k+1} = \frac{1}{2}({\bf X}_k + ({\bf X}_k^{-1})^\top).$$

We use the SVD to write ${\bf A} = {\bf U \: \Sigma \: V^\top}$, and write  ${\bf D}_k = {\bf U}^\top \: {\bf X}_k \: {\bf V}$.\\ Show that ${\bf D}_0 = {\bf \Sigma}$ and ${\bf D}_{k+1} = \frac{1}{2}({\bf D}_k + ({\bf D}_k^{-1})^\top)$.

\subsection{}\label{scheme_diag} From Q\ref{scheme}, each ${\bf D}_k$ is diagonal. If $d_{ki}$ is the $i$-th diagonal element of ${\bf D}_k$, show
$$ d_{(k+1)i} = \frac{1}{2} \left(d_{ki} + \frac{1}{d_{ki}}\right). $$


\subsection{}\label{conv} Show that $d_{\infty i} = 1$ is a fixed point of the scheme described in Q\ref{scheme_diag}.

\subsection{} We assume that $d_{ki} \to d_{\infty i}$ when $k \to \infty$. Show that ${\bf X}_k \to {\bf U \: V}^\top$.

